\subsubsection{Neural MT: Separating hype from the reality}

Machine Translation (MT) has been used by many people for some time now as a productivity tool, with demonstrable success. 
Recently, a new paradigm -- Neural MT (NMT) -- has emerged as a contender to replace statistical MT as the new state-of-the-art in the field. 
While there is no doubt that NMT has enormous potential, we argue that this has been overhyped, a situation which if left unchecked will lead to unrealistic expectations of its capabilities and ultimately to a host of disappointed users. 
We will demonstrate that this has been seen before with the advent of new approaches to MT, and each time this happens it has the potential to undermine the relationship MT developers need to have with translators, the principal users of the technology. 
Despite overblown claims regarding its prospects, we argue -- as we have done for many years now -- that the translation community has little to fear from MT, and that where human input is required, the translator remains the most important link in the chain. 
Indeed, MT developers rely on translators in a number of ways, and if MT is to improve still further, bilateral partnerships need to be formed between both communities for such advances to be made as fast as possible.
